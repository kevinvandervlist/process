\documentclass[]{uva-bachelor-thesis}

% Define this when starting to use the template:
\def\course{Software Process}
\def\assignment{Assignment 2: Scrum and XP}
\def\group{Kevin van der Vlist}
\def\duedate{\today}
% Define your team members hear, each one on a row:
\def\teamtable{\begin{tabular}{ll}
10338152 & kevin.vandervlist@student.uva.nl \\
\end{tabular}}
% The document is now set correctly

\usepackage{graphicx}
\usepackage{wrapfig}
\usepackage{caption}
\usepackage{subcaption}
\usepackage{url}
\usepackage{eurosym}

\usepackage{fancyhdr}
\usepackage[colorlinks, linkcolor=black, urlcolor=black, citecolor=black]{hyperref}
\usepackage{fancyhdr}
\setlength{\headheight}{15pt}

% Chapter and page storage
\def \CurrChapter {}
\def \LastSection {}
\def \CurrSection {}
\newcounter{CurrPage}
\setcounter{CurrPage}{0}

\pagestyle{fancy} 
\renewcommand{\markboth}[2]{\def \CurrChapter {#1}} % For use with \tableofcontents
\renewcommand{\chaptermark}[1]{\def \CurrChapter {#1} \def \CurrSection {}}
\renewcommand{\sectionmark}[1]{ %
  \ifthenelse{ %
    \equal{\thepage}{\value{CurrPage}} %
  } %
  {\def \CurrSection {: #1}} %
  {\let \LastSection \CurrSection \def \CurrSection {: #1} \setcounter{CurrPage}{\value{page}}} %
}

% Headers and footers
\renewcommand{\footrulewidth}{0.5pt}

\fancyhf{}
\fancyfoot[LE,RO]{\textit{\thepage}}
\fancyfoot[RE,LO]{\textit{\group}}
\fancyhead[RE,LO]{\textit{}}
\fancyhead[LE,RO]{ %
  \ifthenelse{ %
    \equal{\thepage}{\value{CurrPage}} %
  } %
  {\textit{\nouppercase{\CurrChapter\LastSection}}} %
  {\let \LastSection \CurrSection \textit{\nouppercase{\CurrChapter\LastSection}}} %
}

\fancypagestyle{plain}{ %
\fancyhf{} %
\fancyfoot[LE,RO]{\textit{\thepage}} %
\fancyfoot[RE,LO]{\textit{\group}} %
\renewcommand{\headrulewidth}{0pt} % remove lines as well
}

\setlength{\parskip}{.5em}

\title{\course}
\subtitle{\assignment}
\author{\group}
\team{\teamtable}

\begin{document}
\maketitle

\setlength{\parskip}{0px}
\tableofcontents
\setlength{\parskip}{.5em}

\clearpage

% What a mess, I'm so fucking tired I can't even keep my eyes open... Damn skiing trip...

\chapter{Scrum and XP}
Make an overview of practices of scrum and xp and list the intent of the practice

\section{Scrum}
\paragraph{Incremental knowledge creation} is a central idea in Scrum. 
Knowledge is acquired by making sure both programmers and domain experts have regular and easy contact.
This also follows the belief that valuable individuals do the best and most effective work when they are given some freedom to do their job.

\paragraph{Sprints} are the time slots that are used for developing a version. 
The focus in Scrum is on release early, release often. 
This allows for fast and worthwile interaction regarding the product of a customer.

\paragraph{The Backlog} is a kind of prioritised list containing the tasks that have yet to be done in a project. 
In each sprint, a certain amount of items is chosen for that sprint. 
These items are picked based on their importance on the backlog. 
The order can vary during the time of a project. 
This makes sure the development team is focusing on items that really matter to the customer.

\paragraph{Daily standup} meetings are important because they make sure everyone in the team knows who is working on what. 
This makes sure there is a common body of knowledge, and impediments ('roadblocks') are discovered early on. 

\section{XP}
\paragraph{Coding} is the most important aspect of XP. 
It is the root of a project, it allows it to exist. 
Effort should be given to code, so it is clear, concise and unambiguous. 

\paragraph{Testing} is also important, as it provides insight to the correctness of code. 
Basically, the more tests the better. 
Categories like acceptance tests (verifying requirements) and unit tests (verifying logical blocks) can be distinguished.

\paragraph{Listening} is about \emph{understanding} the client and the domain. 
Make sure you provide what the other is looking for.

\paragraph{Designing} is important because it allows for the evolution of a project. 
A sophisticated design allows for fluent, easy evolution of the codebase. 
By enforcing designing, long term flexibility is way more likely.

Focus in XP is on the project team. 
Make sure everyone is free and able to learn, and make sure they can work together. 
Make sure that communication is easy, and make sure a lot of feedback is given within the team.

/input{essay.tex}
\chapter{Organisational Resilience and Software Engineering}
% How to apply this knowledge for SE?
\section{Resilience: An introduction}
Resilience is a concept that is quite common in organisational theory. 
There is no single definition, but several of them have been given in existing literature.
Examples are ``the ability of the organisation to simply 'bounce back' from a 'distinctive, discontinuous event that creates vulnerability and requires an unusual response'{''} \cite{JCCM:JCCM558} and ``the ability to 'absorb' strain or change with a minimum of disruption'' \cite{JCCM:JCCM558}.

Important is that the factors contributing to resilience can be developed, measured and managed.
If this is note the case they are external factors, not under the control of an organisation. 
Common factors that can be used to improve resilience are managerial information seeking and continuity planning.
Managerial information seeking is useful because it allows the management to make well-informed decisions, contributing to an organisation built on ``solid ground''.

Resilient organisations really start to shine when they are under stress, be it internally or externally. 
Organisations that have achieved levels of organisational resilience will show behaviour of higher quality \cite{vogus2007organizational}.
This results in either a longer period of being able to withstand the stress or resolving the problems earlier, gaining an edge over the competition that does not.

A resilient organisation usually employs a thorough feedback loop \cite{vogus2007organizational} that employs learning.
This makes sure they can get better over time, by analysing minor problems and indicators showing clues of potential failure. 
Such analyses increase the latent resilience in employees and processes, allowing them to provide an even better response in the next crucial moment.

Individual employees are given the power to identify issues, and decide in the way they think is good. 
As an example, stopping a production process might be costly, but a resilient organisation will empower employees to do this in favour of keeping quiet.
Resilient organisations strive for continuous improvement.

In short: Resilient organisations should identify early warning signals for failure, so they can be robust and flexible.
This in turn allows the organisation to handle strain and stress, gaining an edge over competitors who do not.

\section{Lessons for Software Engineering}
au


%\bibliographystyle{abbrv}
%\bibliography{references}
%\chaptermark{Bibliography}

\end{document}
