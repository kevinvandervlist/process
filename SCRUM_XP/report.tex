\documentclass[]{uva-bachelor-thesis}

% Define this when starting to use the template:
\def\course{Software Process}
\def\assignment{Assignment 2: Scrum and XP}
\def\group{Kevin van der Vlist}
\def\duedate{\today}
% Define your team members hear, each one on a row:
\def\teamtable{\begin{tabular}{ll}
10338152 & kevin.vandervlist@student.uva.nl \\
\end{tabular}}
% The document is now set correctly

\usepackage{graphicx}
\usepackage{wrapfig}
\usepackage{caption}
\usepackage{subcaption}
\usepackage{url}
\usepackage{eurosym}

\usepackage{fancyhdr}
\usepackage[colorlinks, linkcolor=black, urlcolor=black, citecolor=black]{hyperref}
\usepackage{fancyhdr}
\setlength{\headheight}{15pt}

% Chapter and page storage
\def \CurrChapter {}
\def \LastSection {}
\def \CurrSection {}
\newcounter{CurrPage}
\setcounter{CurrPage}{0}

\pagestyle{fancy} 
\renewcommand{\markboth}[2]{\def \CurrChapter {#1}} % For use with \tableofcontents
\renewcommand{\chaptermark}[1]{\def \CurrChapter {#1} \def \CurrSection {}}
\renewcommand{\sectionmark}[1]{ %
  \ifthenelse{ %
    \equal{\thepage}{\value{CurrPage}} %
  } %
  {\def \CurrSection {: #1}} %
  {\let \LastSection \CurrSection \def \CurrSection {: #1} \setcounter{CurrPage}{\value{page}}} %
}

% Headers and footers
\renewcommand{\footrulewidth}{0.5pt}

\fancyhf{}
\fancyfoot[LE,RO]{\textit{\thepage}}
\fancyfoot[RE,LO]{\textit{\group}}
\fancyhead[RE,LO]{\textit{}}
\fancyhead[LE,RO]{ %
  \ifthenelse{ %
    \equal{\thepage}{\value{CurrPage}} %
  } %
  {\textit{\nouppercase{\CurrChapter\LastSection}}} %
  {\let \LastSection \CurrSection \textit{\nouppercase{\CurrChapter\LastSection}}} %
}

\fancypagestyle{plain}{ %
\fancyhf{} %
\fancyfoot[LE,RO]{\textit{\thepage}} %
\fancyfoot[RE,LO]{\textit{\group}} %
\renewcommand{\headrulewidth}{0pt} % remove lines as well
}

\setlength{\parskip}{.5em}

\title{\course}
\subtitle{\assignment}
\author{\group}
\team{\teamtable}

\begin{document}
\maketitle

\setlength{\parskip}{0px}
\tableofcontents
\setlength{\parskip}{.5em}

\clearpage

% What a mess, I'm so fucking tired I can't even keep my eyes open... Damn skiing trip...

\chapter{Scrum and XP}
Make an overview of practices of scrum and xp and list the intent of the practice

\section{Scrum}
\paragraph{Incremental knowledge creation} is a central idea in Scrum. 
Knowledge is acquired by making sure both programmers and domain experts have regular and easy contact.
This also follows the belief that valuable individuals do the best and most effective work when they are given some freedom to do their job.

\paragraph{Sprints} are the time slots that are used for developing a version. 
The focus in Scrum is on release early, release often. 
This allows for fast and worthwile interaction regarding the product of a customer.

\paragraph{The Backlog} is a kind of prioritised list containing the tasks that have yet to be done in a project. 
In each sprint, a certain amount of items is chosen for that sprint. 
These items are picked based on their importance on the backlog. 
The order can vary during the time of a project. 
This makes sure the development team is focusing on items that really matter to the customer.

\paragraph{Daily standup} meetings are important because they make sure everyone in the team knows who is working on what. 
This makes sure there is a common body of knowledge, and impediments ('roadblocks') are discovered early on. 

\section{XP}
\paragraph{Coding} is the most important aspect of XP. 
It is the root of a project, it allows it to exist. 
Effort should be given to code, so it is clear, concise and unambiguous. 

\paragraph{Testing} is also important, as it provides insight to the correctness of code. 
Basically, the more tests the better. 
Categories like acceptance tests (verifying requirements) and unit tests (verifying logical blocks) can be distinguished.

\paragraph{Listening} is about \emph{understanding} the client and the domain. 
Make sure you provide what the other is looking for.

\paragraph{Designing} is important because it allows for the evolution of a project. 
A sophisticated design allows for fluent, easy evolution of the codebase. 
By enforcing designing, long term flexibility is way more likely.

Focus in XP is on the project team. 
Make sure everyone is free and able to learn, and make sure they can work together. 
Make sure that communication is easy, and make sure a lot of feedback is given within the team.

/input{essay.tex}
\chapter{Essay}

Agile methods like Scrum and XP are both based on a few very fundamental assumptions.
They both are very fond of semi-independent, self organising workflows. 
In both processes an assumption is made that the individual is a professional that is able to do its job very well.
So, when a team consists of cunning, socially apt individuals those approaches will probably work very well. 

However, most people are very poor performers. 
And because our field of work is always demanding more people, acceptance interviews are sometimes lacking. 
This results in a lot of people that are not as cunning as one might hope them to be. 
A lack of knowledge or understanding can then result in loss of productivity, or even in a disaster when the amount of skills will deteriorate badly enough inside a team.

So, in order for both Scrum and XP to work, you need a dedicated, professional and cunning team of people.

To what extent would it have solved the problems in the failed project you studied

In the case of the \emph{Betuwelijn}, no real differences would have been made in my opinion. 
Since most of the big failures in the project are related to manipulated numbers from reports, a domain expert would have just behaved in the same way. That person would still be twisting those reports in their favour. 

Another issue in the \emph{Betuwelijn} project was the ever changing scope. Since agile methods even \emph{welcome} this, an even worse outcome might be achieved. The reason I am not sure of this is because in an environment that expects and welcomes change, a more controlled process to these changes might have been started. If this would be the case, I would expect a better outcome. This would be possible because all changes would be critically reviewed by the team and the expert(s) to make sure they make sense.

A trade off that the Agile methods have is their performance. 
While the literature claims high performance gains, empirical studies show that these are almost never achieved, because the balance of all the agile principles need to be met in a perfect manner. This is hardly ever the case. 

Another shortcoming is the lack of direction. 
When a lot of freedom is given to the team to decide on the spot, wrong decisions can be made. A domain expert usually cannot foresee the importance of software related decisions and assumptions. 
This can result in a strained development process, in which a domain expert used to ``just go along'' with a team because he did not understand the magnitude of decisions. 

A last trade off is the lack of out of the box thinking. Domain experts do have a lot of knowledge in their heads, but they also can't use an outside view to approach a problem. 
If a domain expert is very stubborn, a non-optimal solution can be created (see the requirements engineering course).




%\bibliographystyle{abbrv}
%\bibliography{references}
%\chaptermark{Bibliography}

\end{document}
