\documentclass[]{uva-bachelor-thesis}

% Define this when starting to use the template:
\def\course{Software Process}
\def\assignment{Assignment 3: RUP and SCRUM differences}
\def\group{Kevin van der Vlist}
\def\duedate{\today}
% Define your team members hear, each one on a row:
\def\teamtable{\begin{tabular}{ll}
10338152 & kevin.vandervlist@student.uva.nl \\
\end{tabular}}
% The document is now set correctly

\usepackage{graphicx}
\usepackage{wrapfig}
\usepackage{caption}
\usepackage{subcaption}
\usepackage{url}
\usepackage{eurosym}

\usepackage{fancyhdr}
\usepackage[colorlinks, linkcolor=black, urlcolor=black, citecolor=black]{hyperref}
\usepackage{fancyhdr}
\setlength{\headheight}{15pt}

% Chapter and page storage
\def \CurrChapter {}
\def \LastSection {}
\def \CurrSection {}
\newcounter{CurrPage}
\setcounter{CurrPage}{0}

\pagestyle{fancy} 
\renewcommand{\markboth}[2]{\def \CurrChapter {#1}} % For use with \tableofcontents
\renewcommand{\chaptermark}[1]{\def \CurrChapter {#1} \def \CurrSection {}}
\renewcommand{\sectionmark}[1]{ %
  \ifthenelse{ %
    \equal{\thepage}{\value{CurrPage}} %
  } %
  {\def \CurrSection {: #1}} %
  {\let \LastSection \CurrSection \def \CurrSection {: #1} \setcounter{CurrPage}{\value{page}}} %
}

% Headers and footers
\renewcommand{\footrulewidth}{0.5pt}

\fancyhf{}
\fancyfoot[LE,RO]{\textit{\thepage}}
\fancyfoot[RE,LO]{\textit{\group}}
\fancyhead[RE,LO]{\textit{}}
\fancyhead[LE,RO]{ %
  \ifthenelse{ %
    \equal{\thepage}{\value{CurrPage}} %
  } %
  {\textit{\nouppercase{\CurrChapter\LastSection}}} %
  {\let \LastSection \CurrSection \textit{\nouppercase{\CurrChapter\LastSection}}} %
}

\fancypagestyle{plain}{ %
\fancyhf{} %
\fancyfoot[LE,RO]{\textit{\thepage}} %
\fancyfoot[RE,LO]{\textit{\group}} %
\renewcommand{\headrulewidth}{0pt} % remove lines as well
}

\setlength{\parskip}{.5em}

\title{\course}
\subtitle{\assignment}
\author{\group}
\team{\teamtable}

\begin{document}
\maketitle

\setlength{\parskip}{0px}
\tableofcontents
\setlength{\parskip}{.5em}

\clearpage

\chapter{RUP\cite{tenessentials} and SCRUM differences}

\section{SCRUM}
For the complete overview of SCRUM, see the previous document.
In short: It is a very light weight, flexible approach to manage software development. 
Light weight means that is hardly imposes any structure, the only obligations are the three procject roles, sprints and their initialisation / reflection and artefacts like the backlog. 

One of the key components for SCRUM is the freedom it gives to developers. 
They are professionals and they should be accountable for their own work. 

\section{RUP}
The RUP process contains way more structure then SCRUM, but it is still an agile process. 
It consists of more phases, more roles and more obligatory documents. 
The contents of RUP, however, can be seen as a predecessor for SCRUM. 

\section{Differences}
RUP is based on a set of best practices\footnote{From: \url{http://en.wikipedia.org/wiki/IBM_Rational_Unified_Process}}:
\begin{itemize}
  \item Develop iteratively, with risk as the primary iteration driver[2]
  \item Manage requirements
  \item Employ a component-based architecture
  \item Model software visually
  \item Continuously verify quality
  \item Control changes
\end{itemize}

The key part of these practices is to provide a relatively strict approach to managing a development process.
Where SCRUM leaves it up to the developers and the product owner (in case of prioritising), RUP tries to provide a more segmented approach. 

The artefacts used during the process is also tailored to a more strict environment. 
Where a SCRUM team tends to be versatile and should have broad knowledge, a RUP process contains more pointers to specific roles. 
As an example, one of the artefacts can be a complete architectural design, consisting of various views.
In RUP, this is preferably done in advance, so the resemblance at things like this is more towards classical waterfall then current, dynamic SCRUM.

Another difference is the cycle vs sprints approach. 
While SCRUM focuses on having short sprints, RUP is more like waterfall again. 
The cycle has a complete plan, including an end date of the project. 
So, instead of applying sprints ``until happy'', a strict plan (with intermediate checkpoints) is written and adhered. 

\bibliographystyle{abbrv}
\bibliography{references}
\chaptermark{Bibliography}

\end{document}
