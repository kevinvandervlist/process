\chapter{Leprechauns in collecting bug reports}
One of the chapters in the Making Software is called ``The Art of Collecting Bug Reports'' \cite{makingsoftware}.
The core subject of this chapter is what bug reports are captured, and what information they contain.
At the beginning of the chapter, two bug reports are presented, found in the bug database of the Eclipse\footnote{\url{http://www.eclipse.org/}}. 
One of them is shows a perfectly written report, including a minimal test case needed to reproduce the bug, while the other is not even a bug report.
It is more a kind of help request. 

The authors now conclude there is \emph{definitely} a difference in the quality of bug reports, solely based on these two examples.
When verifying the claim, I had a look at the Eclipse bug database myself. 
While querying for a list of invalid bugs\footnote{\url{https://bugs.eclipse.org/bugs/buglist.cgi?query_format=advanced&resolution=INVALID}} a huge amount of bugs was returned.
Only this query already returns at least 10000 bugs.
This claim holds, but could have been supported more thoroughly.

\section{What makes a good bug report}
The chapter then continues to explain the contents of a good bug report. 
This is done based on a paper \cite{Bettenburg:2008:MGB:1453101.1453146} of (among others) the authors of the chapter. 
In this paper, they describe how they approached a set of \emph{experienced} developers from three different large open source projects (Mozilla, Apache and Eclipse).
A questionnaire was created, and both developers and bug reporters had to fill these in.

The selection of these developers and bug reports might be problematic though. 
As an underlying assumption, the authors say that experienced developers have better knowledge on what makes up a good bug report, and what not. 
Even though this might be the case, there might be aspects in bug reports that are of specific importance to new developers, or to developers new to a project. 
By only asking experienced developers within the project, important clues like these might be missed.

Another important assumption regarding the quality of bug reports is that business developers have the same needs as those in open source projects.
However, developers working in an open source project experience different environmental factors then their commercial counterparts. 
The first work for a community backing company, or even as a volunteer, while the latter are employed, and do it for financial compensation. 
They commercial developers also work with business goals, and might prioritise bugs differently based on managerial influence (e.g. being able to sell a product).

As for the bug reporters, some interesting remarks can be made as well. 
The study is presented in the chapter as filled in by reporters of bugs. 
A few sections later, the authors of the chapter conclude there is a mismatch between what bug reporters \emph{know} developers need and what bug reporters \emph{provide} them with. 

This is purely misleading.
In the original paper, they state that ``a number of'' bug reporters explicitly mentioned that they are developers themselves, but in other projects. 
There is no number available, and they were just included with the rest of the bug reporters. 
In other words, among the bug reporters are actually developers. 
This has a huge effect on the research, because it colours the results. 

Of course bug reporters (whom are developers themselves) \emph{know} what they need in a bug report. 
They are aware of the same knowledge, and routinely apply the same task (fixing a bug based on a report). 
To make the discrepancy even worse, bug reporters of open source projects have totally different incentives to submit a bug report at all. 
The only people willing to submit a report will probably be tech-savy persons, which also colours the results.
In a commercial setting, users \emph{have} to employ tools for their daily job, and coming across a bug basically prohibits them from working on a task. 
In order to have it fixed, they can perhaps file a report, or send an email, but this will probably be seen as a necessity, not something they want to do. 

As can be seen, this section raises more questions then it answers.
I don't know the answers, so I can't answer the questions. 
However, I do know that writing the results down in the way as done in the chapter is highly suggestive. 
It indicates a clear mismatch, and therefore the birth of a leprechaun.

\section{Handling bug reports}
A quote from the chapter is ``A very interesting observation is that developers do not suffer too much from bug duplicates, although earlier research considered this to be a problem'', where they cite a specific paper \cite{Anvik:2005:COB:1117696.1117704}.
A closer look on the paper shows that the issue is not correctly identified.
Having duplicates is identified as a problem, but on closer inspection having to \emph{manually identify} the bugs is the actual problem. 
The paper says nothing about having duplicate bugs itself as being a problem. 

Another interesting citation is ``and bug reports with many comments get fixed sooner \cite{Hooimeijer:2007:MBR:1321631.1321639}``.
A look at the paper showed that comments indeed have an effect on the time it takes to fix bugs, but this effect is not linearly (as suggested in the quotation).
The shorter the time between the bug report and the comments, the higher their effect is on the time it takes to fix it. 
The model in the paper has several intervals, leading up to a 4 day period after the opening of a bug report. 
With each interval, the effect is becoming smaller, and after the four days there probably is no significant effect anymore.
This is a different story then the one told in the chapter of making software.

Overall the chapter is well written, but several shortcomings have been identified. 
This does not mean the whole message can be discarded, but use this chapter with caution.
If applied blindly, more leprechauns will come into existence, and myths will be kept in circulation.
