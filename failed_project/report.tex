\documentclass[]{uva-bachelor-thesis}

% Define this when starting to use the template:
\def\course{Software Process}
\def\assignment{Assignment 1: The Betuwelijn}
\def\group{Kevin van der Vlist}
\def\duedate{\today}
% Define your team members hear, each one on a row:
\def\teamtable{\begin{tabular}{ll}
10338152 & kevin.vandervlist@student.uva.nl \\
\end{tabular}}
% The document is now set correctly

\usepackage{graphicx}
\usepackage{wrapfig}
\usepackage{caption}
\usepackage{subcaption}
\usepackage{url}
\usepackage{eurosym}

\usepackage{fancyhdr}
\usepackage[colorlinks, linkcolor=black, urlcolor=black, citecolor=black]{hyperref}
\usepackage{fancyhdr}
\setlength{\headheight}{15pt}

% Chapter and page storage
\def \CurrChapter {}
\def \LastSection {}
\def \CurrSection {}
\newcounter{CurrPage}
\setcounter{CurrPage}{0}

\pagestyle{fancy} 
\renewcommand{\markboth}[2]{\def \CurrChapter {#1}} % For use with \tableofcontents
\renewcommand{\chaptermark}[1]{\def \CurrChapter {#1} \def \CurrSection {}}
\renewcommand{\sectionmark}[1]{ %
  \ifthenelse{ %
    \equal{\thepage}{\value{CurrPage}} %
  } %
  {\def \CurrSection {: #1}} %
  {\let \LastSection \CurrSection \def \CurrSection {: #1} \setcounter{CurrPage}{\value{page}}} %
}

% Headers and footers
\renewcommand{\footrulewidth}{0.5pt}

\fancyhf{}
\fancyfoot[LE,RO]{\textit{\thepage}}
\fancyfoot[RE,LO]{\textit{\group}}
\fancyhead[RE,LO]{\textit{}}
\fancyhead[LE,RO]{ %
  \ifthenelse{ %
    \equal{\thepage}{\value{CurrPage}} %
  } %
  {\textit{\nouppercase{\CurrChapter\LastSection}}} %
  {\let \LastSection \CurrSection \textit{\nouppercase{\CurrChapter\LastSection}}} %
}

\fancypagestyle{plain}{ %
\fancyhf{} %
\fancyfoot[LE,RO]{\textit{\thepage}} %
\fancyfoot[RE,LO]{\textit{\group}} %
\renewcommand{\headrulewidth}{0pt} % remove lines as well
}

\setlength{\parskip}{.5em}

\title{\course}
\subtitle{\assignment}
\author{\group}
\team{\teamtable}

\begin{document}
\maketitle

\setlength{\parskip}{0px}
\tableofcontents
\setlength{\parskip}{.5em}

\clearpage
\chapter{The Betuwelijn}
\section{The project}
The project I have picked for this case is one of the most controversial, Dutch railroad projects.
It is called the \emph{Betuwelijn}. 
According to the Dutch government, the \emph{Betuwelijn} is a ``fast, direct freight train connection between the port of Rotterdam and the European hinterland'' \cite{betuweinfo}.

It has been constructed so increased activities from within the port of Rotterdam can be sustained without putting extra strain on the roads. 
Another benefit is the fact that trains are relatively better for the environment than a line of freight trucks. 

\section{The failure}
Even though the project has been completed, it can not be called successful. 
During preparation and execution of the \emph{Betuwelijn} project, a lot of errors have been made.

The most obvious mistake has been the estimation on how much the project is going to cost. 
When the \emph{Betuwelijn} was first presented in 1990, an estimate of fl. 2.5 billion (or \euro 1.13 billion) was stated \cite{commissie-duivesteijn}.
When the project was finally completed, total cost was \euro 4.8 billion \cite{commissie-duivesteijn}. 
This equals 424.8 \% of the original budget, which is a difference of \euro 3.67 billion.

Because of the huge scale of the project, an ``emergency fund'' was created, so unforeseen events could be dealt with. 
The value of this fund was \euro 985 million. 
When the Dutch House of Representatives (\emph{tweede kamer}) realised this amount of money might be added to the staggering costs of the project, they decided to conduct a thorough research on the subject.
They realised that control on large infrastructural projects is lacking \cite{commissie-duivesteijn}.

\section{Effort to make the project successful}
% What was done to make the project successful
The Dutch government made a choice to enhance the capacity of goods that could be transferred in and out of the port of Rotterdam, especially towards Germany. 
They made a so called ``strategic indicative decision'' \cite{beleidsinformatie-betuweroute}, in order to reach this goal. 
In that same plan, a preference to a railroad project was voiced. 

In order to analyse the viability of the project, a number of studies were being conducted.
The studies were supposed to analyse the effects of the project as future economical effects. 
In order to be sure that the right decision was being made, comparative studies for different means of transport were conducted as well.

Another important decision to ensure a successful project was to make use of the knowledge of the national Dutch railroad company, the \emph{Nederlandse Spoorwegen} (NS).
The government wanted them to prepare the project, so their knowledge of building and maintaining railroads could be reused \cite{beleidsinformatie-betuweroute}.

\section{Reasons for failure}
The report \cite{commissie-duivesteijn} about the \emph{Betuwelijn}, commissioned by Dutch House of Representatives, contains the following reasons for failure.
\begin{itemize}
  \item Costs are structurally underestimated.
  \item Gains are structurally overestimated.
  \item Social complexity (legal issues)
  \item Technical complexity
  \item Financial complexity
\end{itemize}

But there were other mistakes. 
The government started by acting and thinking in terms of ``building'' the \emph{Betuwelijn}, even before a thorough risk analysis had been conducted.
All of the conducted studies appeared to justify a pre-made decision \cite{commissie-duivesteijn}. 
This made discussing it by the representatives impossible, because the decision had already been made.

Lack of communication was also a huge problem. 
Different parties hardly ever communicated. 
Governments and private parties communicating, and even within governmental bodies this sometimes was an issue. 

Another major problem was the opposing interests of the different parties involved in the project \cite{de2012verstrikking}.


The \emph{Rekenkamer}, the Dutch audit office, executed a study dedicated \cite{beleidsinformatie-betuweroute} to the \emph{Betuwelijn} in the year 2000.
They had already recognised problems within the execution of the project, but they hardly acted upon their findings.
Basically, the negative parts of the report were ignored.

% What was the reasons for failure (the technical reason / manifestation, the root cause, and why it was not spotted and contained / mitigated)
\subsection{Root cause}
The root cause for this project is twofold. 
First of all, there is a scoping issue \cite{de2012verstrikking}. 
To compensate to all opposing groups, a lot of concessions were done. 
For example, quite a few changes in favour of the environment were implemented while the project was underway as well as the inclusion of a piece of rail in the port of Rotterdam\footnote{This alone was a change worth more then \euro 1 billion.}. 
This made the project disproportionally more expensive.

The second part of the cause is the way information about the project was handled. 
Individuals in various governmental positions were quite ambitious, so completing such a project felt like achieving something. 
In order to achieve their goal, various reports and studies have been manipulated. 
Manipulation happened by exaggerating the benefits, while ignoring the negative aspects of reports\cite{eenvandaag, commissie-duivesteijn, de2012verstrikking}. 
Another common mistake was comparing different future prospects.
For example: results with prospects beneficial for the railroad project were compared with studies based on negative effects with other transport sectors\cite{commissie-duivesteijn}. 
Researcher Jan Anne Annema (TU Delft), who did similar research about the growth of Schiphol, even calls this \emph{confirmation bias}\cite{huys2009politics, kahneman}, and discovered this happens often.

Arnold Heertje, emeritus professor from the UvA, provides us with another issue \cite{eenvandaag}.
External bureaus are hired by the government to analyse projects.
Because of market constraints\footnote{The same companies usually take on work after a project is approved.} these bureaus tent to echo the opinions of their employer. 
This results in ``we want project X, write a report that shows it's viable.'' like assignments.

\subsection{Why it remained uncovered}
Most of the studies that were executed by order of the government were written based on the perspective that the project should be finished, no matter what. 
A few years after the project started, discussion broke loose again over the necessity of itself. 
During this governmental reconsideration, all studies had been executed from the railroads' perspective again \cite{beleidsinformatie-betuweroute, commissie-duivesteijn}.
Because the project had a special status, 6-monthly progress reports were delivered to the \emph{tweede kamer}. 
These reports were always very positive about the progress of the project \cite{commissie-duivesteijn}.

As soon as people started doubting reports, replies were made containing references to the huge \emph{strategic importance} the project had. 
Critics were told that this project was needed for the future economical growth of regions depending on the transport of goods.
It goes without saying that these references were hardly backed up, and if they were it had been done with sloppily executed studies\cite{de2012verstrikking, eenvandaag}.

The benefits of the project had been significantly exaggerated as well. 
Growth and capacity prognoses, justifying a certain investment level, were quite optimistic. 
In practice, actual rate was way lower. 
Sometimes the indicated rates were 20 years behind their prognoses \cite{beleidsinformatie-betuweroute, commissie-duivesteijn}.

\section{Prevention methodology and solution adequacy}
% what can we learn from that about the adequacy of methods and solutions
People in governmental positions who are responsible for a large infrastructural project need to be aware of their position.
What happens now is that they are unaware of their own ignorance. 
They \emph{think} they do positive things for a community, but private companies are running away with the profits \cite{eenvandaag}.

Tunnel vision, determining that a project \emph{should} and \emph{will} be started is also a danger. 
External, independent auditors should \cite{eenvandaag, de2012verstrikking} should be more critical and pro-active. 
In some controller cases you can even speak about the lack of knowledge \cite{huys2009politics}.
These auditors should also be aware that reports are being manipulated.
If they do, governments can use these independent auditors to objectify facts and figures about large projects \cite{eenvandaag}.

One would think that failures will be of use for further projects, so lessons can be learned from the made mistakes.
According to Arnold Heertje this hardly ever happens \cite{eenvandaag}.

\section{Failure signs}
The most pressing failure signs should have been the ever changing scope and the endless growth of budget. 
When coupled with the fact that critics were basically ignored should have raised some suspicions \cite{beleidsinformatie-betuweroute, commissie-duivesteijn}.
In case of studying the critics and changing scope and budget, the project just went on, and the budget got increased. 

The government also had been lacking during the preparation of the project. 
There had been no independent study by experts regarding the information used as basis for decisions about the project. 
This basic information basically had not been verified\cite{beleidsinformatie-betuweroute, commissie-duivesteijn}.

When the base choice expressing interest for the \emph{Betuwelijn} had been made early on in the project, some extra obligations were implicitly formulated as well. 
For example, even though a rail project had favour, alternatives had to be carefully studied before making the final decision as well. 
The research committee evaluating the process judged that this had not been done securely enough. 
As said before, each report was positively biased towards the project \cite{beleidsinformatie-betuweroute}.

\bibliographystyle{abbrv}
\bibliography{references}
\chaptermark{Bibliography}

\end{document}
