\documentclass[]{uva-bachelor-thesis}

% Define this when starting to use the template:
\def\course{Software Process}
\def\assignment{Assignment 3: Organisational perspectives}
\def\group{Kevin van der Vlist}
\def\duedate{\today}
% Define your team members hear, each one on a row:
\def\teamtable{\begin{tabular}{ll}
10338152 & kevin.vandervlist@student.uva.nl \\
\end{tabular}}
% The document is now set correctly

\usepackage{graphicx}
\usepackage{wrapfig}
\usepackage{caption}
\usepackage{subcaption}
\usepackage{url}
\usepackage{eurosym}

\usepackage{fancyhdr}
\usepackage[colorlinks, linkcolor=black, urlcolor=black, citecolor=black]{hyperref}
\usepackage{fancyhdr}
\setlength{\headheight}{15pt}

% Chapter and page storage
\def \CurrChapter {}
\def \LastSection {}
\def \CurrSection {}
\newcounter{CurrPage}
\setcounter{CurrPage}{0}

\pagestyle{fancy} 
\renewcommand{\markboth}[2]{\def \CurrChapter {#1}} % For use with \tableofcontents
\renewcommand{\chaptermark}[1]{\def \CurrChapter {#1} \def \CurrSection {}}
\renewcommand{\sectionmark}[1]{ %
  \ifthenelse{ %
    \equal{\thepage}{\value{CurrPage}} %
  } %
  {\def \CurrSection {: #1}} %
  {\let \LastSection \CurrSection \def \CurrSection {: #1} \setcounter{CurrPage}{\value{page}}} %
}

% Headers and footers
\renewcommand{\footrulewidth}{0.5pt}

\fancyhf{}
\fancyfoot[LE,RO]{\textit{\thepage}}
\fancyfoot[RE,LO]{\textit{\group}}
\fancyhead[RE,LO]{\textit{}}
\fancyhead[LE,RO]{ %
  \ifthenelse{ %
    \equal{\thepage}{\value{CurrPage}} %
  } %
  {\textit{\nouppercase{\CurrChapter\LastSection}}} %
  {\let \LastSection \CurrSection \textit{\nouppercase{\CurrChapter\LastSection}}} %
}

\fancypagestyle{plain}{ %
\fancyhf{} %
\fancyfoot[LE,RO]{\textit{\thepage}} %
\fancyfoot[RE,LO]{\textit{\group}} %
\renewcommand{\headrulewidth}{0pt} % remove lines as well
}

\setlength{\parskip}{.5em}

\title{\course}
\subtitle{\assignment}
\author{\group}
\team{\teamtable}

\begin{document}
\maketitle

\setlength{\parskip}{0px}
\tableofcontents
\setlength{\parskip}{.5em}

\clearpage

\chapter{Control in an age of empowerment \cite{simons1995control}}
% View of org + how to improve effectiveness
% * What are contributions?

\section{Contribution}
This article is about the way managers can empower their employees to create valuable output, ideas that have business value.
In order to maximise these valuable assets, a certain level of freedom must be given to employees. 
The trade-off between can be a risk, because the freedom can be abused by individuals, damaging the organisation they are representing.

The author describes four ``Levers of Control'' which guide managers in providing freedom, while putting clear borders on around that freedom.
This is done by making sure the boundary of the unacceptable is clear.

Those four levers are:
\begin{itemize}
  \item Beliefs systems - Be very clear about your core objectives.
  \item Boundary systems - Provide clear boundaries on what is \emph{not} acceptable
  \item Diagnostic control systems - Make challenges, high goals which motivate employee.
  \item Interactive control systems - Interact with employees, encourage personal growth and encourage the pushing of boundaries.
\end{itemize}

\section{Relation to Betuwelijn}
% * How was coordination in failed cased? 
% * How did this evolve? 
% * How is this related to failure?
% * How to improve on that?
The failed case, the \emph{Betuwelijn}, was an innovative and technically challenging project.
No real experience had been gathered, so a lot of the projects challenges were new.
In order to complete the project, it would be beneficial to give some freedom to the partners of the project.
An example is the \emph{NS}, which was used to prepare the project because they had experience with managing rail-infrastructural projects.

Another example is the way the Dutch House of Representatives (\emph{tweede kamer}) provided the different governmental bodies with assignments on what to do. 
Each governmental body receives certain ``tasks'', and they basically are responsible for making sure they get their job done.

In other words, each stakeholder of the project was working in its own belief system.
The overarching governmental body, the \emph{tweede kamer}, had different core objectives then each of the underlying stakeholders. 
Because of this, a conflict in objectives could sometimes be found, like ``What's best for the country'' vs. ``We want this railway project because of prestige''.

In case of boundary systems it is even simpler. 
In practice there were hardly any. 
Every stakeholder could basically do as he pleases, just make sure the project will get there. 
The best example of this is the budget growth, with almost 500\% of the original budget.

A good lever has been the diagnostic control system. 
The project was hugely challenging, so highly motivated employees were related to the project. 
I think this is even strengthened by the visibility of the project. 
If completed successfully it would be one enormous showcase.

Lastly, the interactive control system has also been lacking. 
Interaction between the different stakeholders was hard, and only happened when a direct dependency is found. 
Basically, there is no way to make sure all those parties stay in the same line. 

All of this is related to the failure. 
Because communication, boundaries and belief system were different for each stakeholder, they will all develop their own ``subtype'' of the project. 
I think they will get the idea that their way of doing it is correct, and all others are screwing up. 

When the direction of such a huge project is not unified anymore, it will become hard to do right. 
This was increased even more by the ever changing requirements and scope put down by the \emph{tweede kamer}. 
At the end, none of the parties had a really thorough model of the project anymore I think.
Each stakeholder will start to get the feeling he is the only one that's doing it right, after which his way of doing things will be put in stone ever more.

An (in hindsight obvious) improvement would be to make sure all the levers of control are a) acknowledged by the leading stakeholder, the \emph{tweede kamer}, and b) utilised by them. 
At this point, it is not very probable they were even aware of those. 
It can at least not be identified by looking at the way the project was managed.

But, this did not solely constitute to the failure of the project. 
The biggest threats were, like explained in assignment one, the modified information that's used as a basis and the entanglement (Dutch: ``verstrikking'') of the upper echelons during the project.
At a certain stage, it was just unthinkable to let the project fail, no matter what the cost.

So, a number of improvements would have been made possible by using the described levers of control. 
It would also increase the communication, raise awareness for a common goal and make sure the parties will collaborate more and better. 
It can't be said that this will solve the issues of the project though. 

The endeavour has been a new one, raising technical knowledge and moving boundaries on what could be done and what not. 
This required experimentation every now and then, there were basically no real examples or guidelines on how to handle such a project specifically. 
Hardships would still be expected, and the core reasons of failure were not really addressed by these levers of control, because the \emph{tweede kamer} (in view of the levers: the managers) itself has been the prime target to blame for what happened. 

\chapter{Two types of bureaucracy: Enabling and coercive \cite{adler1996two}}

\section{Contribution}
Starts off with two approaches to formalisation of bureaucracy; a negative one (it's annoying and depressing) and a positive one (jobs can give satisfaction, formalisation can increase the work done $\rightarrow$ more satisfaction).

Contingency () theory is divided as well. 
Some say satisfaction increases with low levels of formalisation while doing routine tasks. 
On the other hand, others find the exact opposite, and say high levels of formalism increase satisfaction. 
It suggests that a lot of these influences can be explained by taking into account that the attitude towards formalism is how ``good'' or ``bad'' rules are experienced. 
The more negative the rule is experienced, the less satisfaction there can be.
Primary goal of paper: develop a theory on how employees do this distinction.

Gouldner: 3 types of rules. 
\emph{Representative}: interest of managers and workers, \emph{punishment}: legitimising actions of one versus other and \emph{mock}: ignored by everyone.

From technology: formalisation is sometimes just solidifying existing structure.
Tools can be designed to lower skills (less dependent on expensive labour $\rightarrow$ automatising) or enhancing existing skills (maximising potential).
Basic debate: user is source of errors vs source of skill and potential.
The latter can be leveraged to increase products and procedures, for example by ``outsourcing'' basic maintenance to end users (Xerox example -- basic copier maintenance).

Parallel: rules can be designed to effectively deal with a shitstorm, not only to optimise process. 
Enabling bureaucracy can be compared to codifying best practices learned over the years in an organisation.
It's coercive counterpart on the other hand is designed to bring out behaviour when people are unwilling. 

Features of enabling formalisation are repair, internal transparency, global transparency and flexibility. 
\emph{Repair} is coercive, or deskilling, and results in lack of trust. 
Used when opportunities of individual employees are higher then the added value of organisation. 
Employee initiative is ignored and will gradually decrease, because it does not feel anything positive of the rules. 

\emph{Internal transparency} from an enabling perspective is when a subject of a procedure can identify it's core components and can act upon failure or malfunctions. 
It is not instantiated to deskill users, but to leverage their potential. 
When seen from a coercion background, its a facade, with ``flat assertions of duties''. 

\emph{Global transparency} is reduced to information sharing / hiding within organisation. 
Again, for defending info or enabling it to use as value-increasing variable.

\emph{Flexibility} is divided again. 
In a coercive world, a sequence is enforced. Suggestions and the optional taking of control in enabling counterpart. 

Again, coercive organisation basically ignores employees and thinks they are stupid. 
Enabling is the opposite, and uses changing conditions or procedures to optimise their process. 

Topology of an organisation can be put in four categories:
\begin{itemize}
\item Organic: low formalised and enabling
\item Autocratic, lowly formalised and coercive
\item Enabling bureaucracy, Highly formalised but enabling
\item Mechanistic: Highly formalised and coercive 
\end{itemize}

This model has a lot of advantages, which all boil down to common sense. 
In two-side model, it is always exaggerating. 
With these four choices, more nuanced divisions can be applied\footnote{No shit sherlock...}. 
It allows to be both formalised and promote employee initiative. 

Enabling bureaucracy apparently is more beneficial. 
In basically all studies presented, this leads to improvements. 
If done correctly, tooling with enabling in mind tends to create a positive closed loop, in which empowerment level keeps rising. 

\section{Relation to Betuwelijn}
% * How was coordination in failed cased? 
% * How did this evolve? 
% * How is this related to failure?
% * How to improve on that?


\bibliographystyle{abbrv}
\bibliography{references}
\chaptermark{Bibliography}

\end{document}
