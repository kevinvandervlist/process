\section{Scrum and knowledge creation \cite{van2012scrum}}
\subsection{Notes}
The central topic in this thesis is knowledge creation in relation with a Scrum workflow.
The following section will contain notes regarding Scrum.
A lot of info about knowledge creation, indexing and usage is in the thesis, but it is not relevant for this assignment.

Scrum uses an agile methodology. 
It is focused on small teams, which are self organising, communicate informally and deal with mostly tacit knowledge. 

Special roles are for the product owner, who has the domain knowledge, and scrum master. 
The latter is part of the development and manages the scrum process.
This process consists of having short sprints, in which possible releases are made in 2-3 weeks. 

Each sprint is structured by getting tasks from a product backlog, which is basically a prioritised, time-indexed todo list. 
These items will then be implemented within the sprint.
Missing knowledge in the team will be acquired on the go. 

Knowledge sharing within the team and product owner can happen in a ba, a shared context.
The form of this shared context is not fixed, and can take any form.

User stories that are the result of ba can be given an estimate on implementation cost by doing planning poker.
In this 'game', a weight is expressed by each individual of the group.
This weight is how difficult it is to implement a certain feature. 
The extremes in estimates (both high and low) can be discussed, after which another estimate is made by the group. 
This is done until an average is reached. 

% Relevant notes

\subsection{Questions}
% Relevant questions
\begin{itemize}
  \item What is Scrum?
  \item Why can there be a gap between having and using knowledge while developing software?
  \item How can domain knowledge of team members be assessed?
  \item What can we do to prevent focusing primarily on usability instead of technical issues?
\end{itemize}
