\chapter{Measuring Resilience Potential: An Adaptive Strategy for Organizational Crisis Planning \cite{somers2009measuring}}
\section{Notes}
% Relevant notes
This paper is about a study which aims to identify which crisis planning factors have a causal relationship to effective adaptive behaviour in crisis (specially focused on the public domain).
According to the paper, existing work shows there is debate on what contributes positively and what not towards this behaviour.
This also applies to the issue of defining the term resilience. 
The definition seems to vary on a per-discipline basis.

In organisational theory it is usually defined as ``the ability of the organisation to simply 'bounce back' from a 'distinctive, discontinuous event that creates vulnerability and requires an unusual response'{''}, or ``the ability to 'absorb' strain or change with a minimum of disruption''.
Another definition is ``Capacity to cope with unanticipated dangers after they have become manifest''.

Other researchers differentiate between active and passive resilience. 
Passive is being able to provide a mere response, active seeks to become better.
Another requisite towards resilience is that the factors can be ``developed, measured and managed''. 

The paper provides a tool that measures if six common factors will influence resilience of an organisation. 
\begin{itemize}
\item Level of perceived risk: influence below statistical significance.
\item Managerial information seeking: correlates positively to resilience. A necessity is that information is interpreted so core content can be extracted and applied. Seeking information in more sources makes the process more time consuming, but correlates with more commitment.
\item Organisational structure (number of managerial layers): No statistically significant results. 
\item Continuity planning: planning improves resilience potential.
\item Participation in community planning: Probably a positive influence, but cannot be proved.
\item Department accreditation (certification): Too few results to formally conclude anything, but appears to imply a positive effect.
\end{itemize}

The paper challenges the idea that a step-by-step manual is a preferred outcome of a crisis planning. 
Instead, internal processes and organisational structures are preferred if they build latent resilience. 
This allows to employ behaviour when the system is under stress.

\section{questions}
% relevant questions
\begin{itemize}
  \item Can you use a tool to measure organisational resilience?
  \item What is organisational resilience?
\end{itemize}

