\chapter{Structure in 5's: A Synthesis of the Research on Organization Design \cite{structure-5}}
\section{Notes}
% Relevant notes
Mintzberg identifies 5 organisation setups, and describes these in his paper. 
They can be summarised as:
\begin{itemize}
\item Simple structure -- Informal, tend to occur in new businesses
\item Machine bureaucracy -- Formalised workflow, highly specialised routine tasks
\item Professional bureaucracy -- Hiring and autonomous usage of professionals
\item Diversified form -- Kind of layered approach with central steering core.
\item Adhocracy -- Low formalisation in order to maximise professional potential and stimulation of new ideas. Can be focused externally (consulting) and internally (research). 
\end{itemize}

Labour within an organisation can be divided in several tasks as well:
\begin{itemize}
  \item Direct Supervision -- Someone gives direct orders.
  \item Standardisation of workprocesses -- Formalised, follow procedures and routines to do the job.
  \item Standardisation of outputs -- The same as above, but focused on production output.
  \item Standardisation of skill -- Educational background of individuals is brought to a standard level. Trainings etc. 
  \item Mutual Adjustment -- Self organising, by doing adhoc communication.
\end{itemize}

\section{Questions}
% Relevant questions
\begin{itemize}
  \item What organisational structures can be identified in a business?
  \item How can you organise this at an employee level?
\end{itemize}

\section{Review}
% Optional: review why fellow students have to read it
It contains the base of the work of Mintzberg.
