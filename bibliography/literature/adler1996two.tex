\chapter{Two types of bureaucracy \cite{adler1996two}}
\section{Notes}
% Relevant notes
Starts off with two approaches to formalisation of bureaucracy; a negative one (it's annoying and depressing) and a positive one (jobs can give satisfaction, formalisation can increase the work done $\rightarrow$ more satisfaction).

Contingency () theory is divided as well. 
Some say satisfaction increases with low levels of formalisation while doing routine tasks. 
On the other hand, others find the exact opposite, and say high levels of formalism increase satisfaction. 
It suggests that a lot of these influences can be explained by taking into account that the attitude towards formalism is how ``good'' or ``bad'' rules are experienced. 
The more negative the rule is experienced, the less satisfaction there can be.
Primary goal of paper: develop a theory on how employees do this distinction.

Gouldner: 3 types of rules. 
\emph{Representative}: interest of managers and workers, \emph{punishment}: legitimising actions of one versus other and \emph{mock}: ignored by everyone.

From technology: formalisation is sometimes just solidifying existing structure.
Tools can be designed to lower skills (less dependent on expensive labour $\rightarrow$ automatising) or enhancing existing skills (maximising potential).
Basic debate: user is source of errors vs source of skill and potential.
The latter can be leveraged to increase products and procedures, for example by ``outsourcing'' basic maintenance to end users (Xerox example -- basic copier maintenance).

Parallel: rules can be designed to effectively deal with a shitstorm, not only to optimise process. 
Enabling bureaucracy can be compared to codifying best practices learned over the years in an organisation.
It's coercive counterpart on the other hand is designed to bring out behaviour when people are unwilling. 

Features of enabling formalisation are repair, internal transparency, global transparency and flexibility. 
\emph{Repair} is coercive, or deskilling, and results in lack of trust. 
Used when opportunities of individual employees are higher then the added value of organisation. 
Employee initiative is ignored and will gradually decrease, because it does not feel anything positive of the rules. 

\emph{Internal transparency} from an enabling perspective is when a subject of a procedure can identify it's core components and can act upon failure or malfunctions. 
It is not instantiated to deskill users, but to leverage their potential. 
When seen from a coercion background, its a facade, with ``flat assertions of duties''. 

\emph{Global transparency}



Of coercive: 

\section{questions}
% relevant questions
\begin{itemize}
  \item Foo
\end{itemize}

\section{Review}
% Optional: review why fellow students have to read it
