\chapter{Control in an Age of Empowerment \cite{simons1995control}}
\section{Notes}
% Relevant notes
This article is about the way managers can empower their employees to create valuable output, ideas that have business value.
In order to maximise these valuable assets, a certain level of freedom must be given to employees. 
The trade-off between can be a risk, because the freedom can be abused by individuals, damaging the organisation they are representing.

The author describes four ``Levers of Control'' which guide managers in providing freedom, while putting clear borders on around that freedom.
This is done by making sure the boundary of the unacceptable is clear.

Those four levers are:
\begin{itemize}
  \item Beliefs systems - Be very clear about your core objectives.
  \item Boundary systems - Provide clear boundaries on what is \emph{not} acceptable
  \item Diagnostic control systems - Make challenges, high goals which motivate employee.
  \item Interactive control systems - Interact with employees, encourage personal growth and encourage the pushing of boundaries.
\end{itemize}

\section{questions}
% relevant questions
\begin{itemize}
  \item How to ``guide'' the empowerment of
\end{itemize}

\section{Review}
% Optional: review why fellow students have to read it
