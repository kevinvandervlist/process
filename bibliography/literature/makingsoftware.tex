\chapter{Making Software \cite{makingsoftware}}
\section{Chapter 22: The evidence for design patterns}
\subsection{Notes}
% Relevant notes
Design patterns are a standard approach to solving a generic problem. 
It is like a algorithm from a text book. 
The general approach is given, but the implementation must still be tailored to the used language, framework and case. 

The added value of design patterns lies in its cognitive value. 
When a person is aware of the pattern, the meaning of it can be reduced to a single chunk (see Miller's paper). 
The rest of the short term memory is therefore still available for use.

Because the above is based on the assumption of \emph{knowing} the pattern, a warning is in place. 
When users are not aware of the patterns, they tend to increase the cognitive load of the approach. 
Another pitfall is the risk of over-engineering. 
People might become a bit overenthusiastic and start to complicate simple cases by enforcing a set of patterns which might not even be needed.

The rest of the chapter contains a lot of references on the statistical guarantees of studies regarding the use of patterns. 
Things like how to do a proper setup, prevent the learning effect and making sure research is statistically significant are mentioned as well.

\subsection{Questions}
% Relevant questions
\begin{itemize}
  \item Why to apply design patterns?
  \item Are design patterns effective?
  \item How to do a comparative research to applying design patterns?
\end{itemize}

\subsection{Review}
% Optional: review why fellow students have to read it
Read this chapter if you want to know whether design patterns can be effective or if they are a class of over-engineering.
The research to answer this question is quite extensive, and in this chapter a nice summary is presented. 
It also provides some interesting pointers if you want to do a study on something related, and statistical significance is of great importance.
Finally, some references regarding cognitive psychology might also be of interest.
