\chapter{Making Software \cite{makingsoftware}}
\section{Chapter 22: The evidence for design patterns}
\subsection{Notes}
% Relevant notes
Design patterns are a standard approach to solving a generic problem. 
It is like a algorithm from a text book. 
The general approach is given, but the implementation must still be tailored to the used language, framework and case. 

The added value of design patterns lies in its cognitive value. 
When a person is aware of the pattern, the meaning of it can be reduced to a single chunk (see Miller's paper). 
The rest of the short term memory is therefore still available for use.

Because the above is based on the assumption of \emph{knowing} the pattern, a warning is in place. 
When users are not aware of the patterns, they tend to increase the cognitive load of the approach. 
Another pitfall is the risk of over-engineering. 
People might become a bit overenthusiastic and start to complicate simple cases by enforcing a set of patterns which might not even be needed.

The rest of the chapter contains a lot of references on the statistical guarantees of studies regarding the use of patterns. 
Things like how to do a proper setup, prevent the learning effect and making sure research is statistically significant are mentioned as well.

\subsection{Questions}
% Relevant questions
\begin{itemize}
  \item Why to apply design patterns?
  \item Are design patterns effective?
  \item How to do a comparative research to applying design patterns?
\end{itemize}

\subsection{Review}
% Optional: review why fellow students have to read it
Read this chapter if you want to know whether design patterns can be effective or if they are a class of over-engineering.
The research to answer this question is quite extensive, and in this chapter a nice summary is presented. 
It also provides some interesting pointers if you want to do a study on something related, and statistical significance is of great importance.
Finally, some references regarding cognitive psychology might also be of interest.


\section{Chapter 07: Why Is It So Hard to Learn to Program?}
\subsection{Notes}
% Relevant notes
Apparently a large difference between natural language and ``foreign'' program language. 
A lot of people fail CS1, but when focusing on core concepts students are apt. 
Several research groups have tried other approaches, like visual programming (increased cognitive load) and ``practical, media oriented'' (worked very well. Significantly highers passing ratings of courses).
No definitive proof, but media computation tends to provide enough context so students are able to ``get'' core programming concepts easier. 

\subsection{Questions}
% Relevant questions
\begin{itemize}
  \item Why do a lot of people struggle while starting programming?
\end{itemize}

\subsection{Review}
% Optional: review why fellow students have to read it
Interesting for some nice background on why learning to program is often experienced as ``abstract'', and ``not suited for me''. 

\section{Chapter 15: Quality Wars: Open Source Versus Proprietary Software}
\subsection{Notes}
% Relevant notes
Hardly any really comparing research exists, because companies are usually not willing to provide source for such analyses.
The author of the chapter took matter into his own hands, and defined a large list of metrics on which four important projects could be compared. 
Relevance of the metrics will not be discussed here, see all I've learned during Software Evolution. 

The four used projects are an academic release of the Windows, OpenSolaris, Linux and FreeBSD.
More specifically, it always was the kernel of  those systems. 
These projects are of comparable size and complexity (except for missing drivers in the case of windows) so they can be compared. 

A huge list of metrics is generated, and overall there is no winner or loser. 
Each project had focus on different aspects of the code quality, and that can be seen. 
None of them were bad, the managerial approaches just differed. 

\subsection{Questions}
% Relevant questions
\begin{itemize}
  \item Can you say that open source (kernel) code quality is higher then proprietary counterpart?
\end{itemize}

\subsection{Review}
% Optional: review why fellow students have to read it
Read this chapter if you are interested in metrics of huge projects.
It's an interesting read, but the conclusions are not shocking.

\section{Chapter xx: }
\subsection{Notes}
% Relevant notes

\subsection{Questions}
% Relevant questions
\begin{itemize}
  \item Foo
\end{itemize}

\subsection{Review}
% Optional: review why fellow students have to read it

\section{Chapter xx: }
\subsection{Notes}
% Relevant notes

\subsection{Questions}
% Relevant questions
\begin{itemize}
  \item Foo
\end{itemize}

\subsection{Review}
% Optional: review why fellow students have to read it

\section{Chapter xx: }
\subsection{Notes}
% Relevant notes

\subsection{Questions}
% Relevant questions
\begin{itemize}
  \item Foo
\end{itemize}

\subsection{Review}
% Optional: review why fellow students have to read it
