\chapter{The Leprechauns of Software Engineering \cite{leprechaun}}
The content below regards chapters 1 up to 5.
\section{Notes}
% Relevant notes
The author is dissatisfied with the amount of false claims the software engineering profession makes.
Instead of ignoring or complaining, the author investigates these to enlighten our profession.

This book starts off with describing the items that usually constitute to creating a software engineering myth. 
Most common mistakes are claims of results that are not described in a paper referenced as source. 
Another is the insecure citations that enhance opinions or statements to an ``empirical research'' status, after which erroneous references to the source mistake are made.

Mistakes like the ones described above are considered a negative force in the software engineering profession. 
The author takes the reader by the hand and shows how to ``debunk'' a myth step by step. 
This process can be summarised in the following steps:
\begin{itemize}
  \item[1] Gather claims
  \item[2] Analyse referenced sources for the claim
  \item[3] a) If analysis is inconclusive, and more references are given, see 2cc
  \item[3] b) If analysis is correct, the claim is proven
  \item[3] c) If analysis is wrong, the claim is disproofed
  \item[4] Write up your findings.
\end{itemize}


\section{questions}
% relevant questions
\begin{itemize}
  \item Why is Software Engineering so full of myths?
  \item How can myths be identified?
  \item What can we do to prevent them?
\end{itemize}
