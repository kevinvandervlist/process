\chapter{Organizational resilience: towards a theory and research agenda \cite{vogus2007organizational}}
\section{Notes}
% Relevant notes
The authors of this paper create an outline for future organisational resilience research. 
It is increasingly important that they investigate why some organisations thrive under pressure, while others crumble.
Existing organisational theory does not answer these questions according to the authors.

Resilience is ``the maintenance of positive adjustment under challenging conditions such that the organisation emerges from these conditions strengthened and more resourceful'' according to the authors.
This includes external events and ongoing strain. 
Another remark is that it can be grown over time.
Competence in one period allows for higher competence in the following periods as well.
It exists independent of other learning capabilities, but in itself relies on past and future learning experiences.

Successful organisations know they operate on performance limits and continuously look for deviations in order to improve their process.
Unexpected events are discovered sooner, quick recovery via established routines is preferred over error avoidance. 
The organisations know they eventually end up making a mistake anyway.

Resilient organisations assume their process continuously needs updating to become better, even when no failures are presented.
Brittle organisations are the opposite and see failure as a confirmation of success.
They also require ``more proof'' in case something occurs, interpreting it as a one-off.

Resilient organisations are also unafraid of taking drastic measures, such as stopping production. 
This allows speaking up on possible problems, without being blamed in extreme ways.
They also think they can deal with every kind of situation that can occur, so they continuously strive for improvement.
This in turn allows for early detection of warning signals because they are used to identify them.
Occurrences of ``near misses'' are also identified as learning moments for an organisation, since some small failures were identified.

\section{questions}
% relevant questions
\begin{itemize}
  \item What are key aspects of a resilient organisation?
  \item Which events or moments in an organisation can be leveraged as learning moments to increase resilience?
\end{itemize}

