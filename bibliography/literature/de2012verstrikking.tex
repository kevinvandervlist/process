\chapter{Verstrikking bij railinfrastructurele projecten \cite{de2012verstrikking}}
\section{Notes}
% Relevant notes
A Master's thesis of Planologie investigating, among others, the \emph{Betuwelijn} project.
I only read the piece on the Betuwelijn.
The reader is given a nice overview on the project.
It starts off with the planning process, describing what state the project was in over time.

A discussion follows explaining all the different pieces of the plan that have been presented to the Dutch House of Representatives (\emph{tweede kamer}) over time.
An overview of the different stakeholders is given, including a listing of there specific wishes.

The huge costs related to the project are also presented. 
For each budget increase, the date and cause of this event are given. 

The last, harsh conclusion is, what the author calls, \emph{trapping}.
This means that individuals cannot take their losses when a project is failing, and that continuous effort is made to save it. 
Raising the investments is one of these continuous efforts.

\section{Questions}
% Relevant questions
\begin{itemize}
  \item What were the different stakeholders within the Betuwelijn?
  \item What were their incentives?
  \item What organisational stages has the Betuwelijn been through?
\end{itemize}

\section{Review}
% Optional: review why fellow students have to read it
This thesis is focused on trapping within railinfrastructural projects. 
Trapping means that stakeholders are taking losses with the project, but do not want to let go. 
Against better judgement, they still want to continue the project to try to salvage it for a bit. 
One of the projects that has been analysed is the Betuwelijn. 
It provides an clear overview of the different stakeholders and there incentives. 
A very clear timescale is given as well.
