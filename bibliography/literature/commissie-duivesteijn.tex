\chapter{Onderzoek naar infrastructuurprojecten \cite{commissie-duivesteijn}}
\section{Notes}
% Relevant notes

In order to analyse the complete process regarding building, managing, financing and using the \emph{Betuwelijn}, a research committee has been appointed. 
The Dutch House of Representatives (\emph{tweede kamer}) is the governmental body that appointed this committee in 2003. 

This report analyses the decisions that outline the complete process. 
During the executed research, the committee claims that there are structural deficiencies regarding the failure of the project.

It starts of with the process leading up to the decision of building the Betuwelijn. 
Alternatives have not been analysed in a correct, defensible way. 
Reports were either prejudiced or they have been selectively used so only favourable aspects are highlighted.

The report does \emph{not} emphasise the way the project has been managed. 
It does mention things, but it is not the central topic.
A core subject of this report is the prioritising of large infrastructural projects by the House of Representatives. 

Another core component is the way the house of representatives is involved in the process. 
The government should have acted responsibly, well informed and should make well-thought decisions. 
This report shows these aspects have definitely been lacking. 

The report concludes with a few proposals on how the House of Representatives can change law and regulations to prevent future procedural failures related to the same causes. 

\section{Questions}
% Relevant questions
\begin{itemize}
  \item What mistakes were made during the preparation of the \emph{Betuweroute} project?
  \item What can the government do to prevent these mistakes in the future?
  \item Why were alternative approaches not given a fair chance?
  \item Why were the costs continuously increasing?
\end{itemize}

\section{Review}
% Optional: review why fellow students have to read it
This is the official research document regarding the Betuwelijn. 
This exact document is used by the government to assess the process.
In order to get a good overview of what happened during all these years, this is a must read.

The research committee has done a great job in analysing the process and documenting all its shortcomings. 
While reading (parts) of it, I did not get the idea that it was a biased document. 
The critique towards the government was relatively harsh, but very concrete approaches to improvements have been given as well.
