\chapter{Organisational Resilience and Software Engineering}
% How to apply this knowledge for SE?
\section{Resilience: An introduction}
Resilience is a concept that is quite common in organisational theory. 
There is no single definition, but several of them have been given in existing literature.
Examples are ``the ability of the organisation to simply 'bounce back' from a 'distinctive, discontinuous event that creates vulnerability and requires an unusual response'{''} \cite{JCCM:JCCM558} and ``the ability to 'absorb' strain or change with a minimum of disruption'' \cite{JCCM:JCCM558}.

Important is that the factors contributing to resilience can be developed, measured and managed.
If this is note the case they are external factors, not under the control of an organisation. 
Common factors that can be used to improve resilience are managerial information seeking and continuity planning.
Managerial information seeking is useful because it allows the management to make well-informed decisions, contributing to an organisation built on ``solid ground''.

Resilient organisations really start to shine when they are under stress, be it internally or externally. 
Organisations that have achieved levels of organisational resilience will show behaviour of higher quality \cite{vogus2007organizational}.
This results in either a longer period of being able to withstand the stress or resolving the problems earlier, gaining an edge over the competition that does not.

A resilient organisation usually employs a thorough feedback loop \cite{vogus2007organizational} that employs learning.
This makes sure they can get better over time, by analysing minor problems and indicators showing clues of potential failure. 
Such analyses increase the latent resilience in employees and processes, allowing them to provide an even better response in the next crucial moment.

Individual employees are given the power to identify issues, and decide in the way they think is good. 
As an example, stopping a production process might be costly, but a resilient organisation will empower employees to do this in favour of keeping quiet.
Resilient organisations strive for continuous improvement.

In short: Resilient organisations should identify early warning signals for failure, so they can be robust and flexible.
This in turn allows the organisation to handle strain and stress, gaining an edge over competitors who do not.

\section{Lessons for Software Engineering}
au
