\chapter{Organisational Resilience and Software Engineering}
% How to apply this knowledge for SE?
\section{Resilience: An introduction}
Resilience is a concept that is quite common in organisational theory. 
There is no single definition, but several of them have been given in existing literature.
Examples are ``the ability of the organisation to simply 'bounce back' from a 'distinctive, discontinuous event that creates vulnerability and requires an unusual response'{''} \cite{JCCM:JCCM558} and ``the ability to 'absorb' strain or change with a minimum of disruption'' \cite{JCCM:JCCM558}.

Important is that the factors contributing to resilience can be developed, measured and managed.
If this is note the case they are external factors, not under the control of an organisation. 
Common factors that can be used to improve resilience are managerial information seeking and continuity planning.
Managerial information seeking is useful because it allows the management to make well-informed decisions, contributing to an organisation built on ``solid ground''.

Resilient organisations really start to shine when they are under stress, be it internally or externally. 
Organisations that have achieved levels of organisational resilience will show behaviour of higher quality \cite{vogus2007organizational}.
This results in either a longer period of being able to withstand the stress or resolving the problems earlier, gaining an edge over the competition that does not.

A resilient organisation usually employs a thorough feedback loop \cite{vogus2007organizational} that employs learning.
This makes sure they can get better over time, by analysing minor problems and indicators showing clues of potential failure. 
Such analyses increase the latent resilience in employees and processes, allowing them to provide an even better response in the next crucial moment.

Individual employees are given the power to identify issues, and decide in the way they think is good. 
As an example, stopping a production process might be costly, but a resilient organisation will empower employees to do this in favour of keeping quiet.
Resilient organisations strive for continuous improvement.

In short: Resilient organisations should identify early warning signals for failure, so they can be robust and flexible.
This in turn allows the organisation to handle strain and stress, gaining an edge over competitors who do not.

\section{Lessons for Software Engineering}
The software engineering profession can learn a number of lessons regarding resilience in software projects.
Our projects are notorious for failures, usually after spending lots of money over a long period of time. 

On the one hand there are the projects in which management is unaware of the direction the project is going. 
They think the project is under control, but this happens to be a false assumption when the first deliverable is presented to a customer. 
Another common scenario is that a lot of people \emph{know} a project is heading towards a failure, but somehow nobody acts accordingly. 
Deadlines fly past, budgets overrun and still nobody takes action.

Such issues can be seen as the early warning signals of a project failure, something indicating the loss of control of the project. 
In resilient organisations, people are trained to detect and then \emph{act} when these signals are discovered. 
This allows for early detection of a project on the loose, which in turn allows for corrections so the project can be put back on track.

Another strength is the power put with individuals. 
In a resilient organisation, any employee can ``stop production'' when sees a defect which needs to be addressed. 
In software engineering, this usually is not the case.

However, combining this with the agile approach -- where the individuals team members have a lot of power -- can result in interesting scenarios. 
For example, say a developer discovers a huge security risk in the codebase of the project.
If time pressure is high, the project can be shipped with the security leak still in the code. 
The priority of the team was at shipping, not fixing a leak which \emph{might} result in abuse.

If a developer had the power to put the project release on hold for a small timespan, effort could be put into fixing the issue. 
The project could then be pushed to the market a moment later, but without a leak wreaking potential havoc.

The same could happen in normal scenarios. 
If developers have the opportunity to pay of future loans on maintainability and security, the future of the product as well as the team are better of. 
This shows an overlap with the feedback loop in organisations, where resilience is a kind of self enforcing loop.
Over time, the team gets better because of both a higher flexibility and a better codebase.

Another lesson learned from resilient organisations is that (managerial) information seeking pays off. 
Translated to the software domain: investing time in what you \emph{need} to know will pay off.
For example, think about finding the actual requirements of a project or investing time in creating an executable architecture to proof your linear scaling techniques are actually viable. 
This reduces both the future risk and cost, while at the same time learning so future deviations of schedule and estimates in general will be easier and more on the spot.

A different lesson learned is the value of redundancy \cite{zhang2010principle}.
Making sure that crucial parts of both a project and a product are redundant allows you to not have a single point of failure. 
As a result, your performance tends to be higher.

