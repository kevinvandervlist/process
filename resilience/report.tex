\documentclass[]{uva-bachelor-thesis}

% Define this when starting to use the template:
\def\course{Software Process}
\def\assignment{Assignment 6: Resilience}
\def\group{Kevin van der Vlist}
\def\duedate{\today}
% Define your team members hear, each one on a row:
\def\teamtable{\begin{tabular}{ll}
10338152 & kevin.vandervlist@student.uva.nl \\
\end{tabular}}
% The document is now set correctly

\usepackage{graphicx}
\usepackage{wrapfig}
\usepackage{caption}
\usepackage{subcaption}
\usepackage{url}
\usepackage{eurosym}

\usepackage{fancyhdr}
\usepackage[colorlinks, linkcolor=black, urlcolor=black, citecolor=black]{hyperref}
\usepackage{fancyhdr}
\setlength{\headheight}{15pt}

% Chapter and page storage
\def \CurrChapter {}
\def \LastSection {}
\def \CurrSection {}
\newcounter{CurrPage}
\setcounter{CurrPage}{0}

\pagestyle{fancy} 
\renewcommand{\markboth}[2]{\def \CurrChapter {#1}} % For use with \tableofcontents
\renewcommand{\chaptermark}[1]{\def \CurrChapter {#1} \def \CurrSection {}}
\renewcommand{\sectionmark}[1]{ %
  \ifthenelse{ %
    \equal{\thepage}{\value{CurrPage}} %
  } %
  {\def \CurrSection {: #1}} %
  {\let \LastSection \CurrSection \def \CurrSection {: #1} \setcounter{CurrPage}{\value{page}}} %
}

% Headers and footers
\renewcommand{\footrulewidth}{0.5pt}

\fancyhf{}
\fancyfoot[LE,RO]{\textit{\thepage}}
\fancyfoot[RE,LO]{\textit{\group}}
\fancyhead[RE,LO]{\textit{}}
\fancyhead[LE,RO]{ %
  \ifthenelse{ %
    \equal{\thepage}{\value{CurrPage}} %
  } %
  {\textit{\nouppercase{\CurrChapter\LastSection}}} %
  {\let \LastSection \CurrSection \textit{\nouppercase{\CurrChapter\LastSection}}} %
}

\fancypagestyle{plain}{ %
\fancyhf{} %
\fancyfoot[LE,RO]{\textit{\thepage}} %
\fancyfoot[RE,LO]{\textit{\group}} %
\renewcommand{\headrulewidth}{0pt} % remove lines as well
}

\setlength{\parskip}{.5em}

\title{\course}
\subtitle{\assignment}
\author{\group}
\team{\teamtable}

\begin{document}
\maketitle

\setlength{\parskip}{0px}
\tableofcontents
\setlength{\parskip}{.5em}

\clearpage

%% Find and read one journal paper about resilience / organizational vigilance. 
%% (Key words: early warning signals, sense making, sense breaking)
%% Find and read one paper that discusses this for software development project.
%% Update your annotated bibliography with interesting facts and thoughts from the papers.



%% The body of research is usually valid for high reliability organizations. These tend to be highly organized, with lot of money spent on problem prevention and training. 
%% These organizations are quite costly and all but lean. 
%% They have the luxury and tendency to be conservative, cautious and accept only high quality solutions. 

%% In software projects it's another world. 
%% There is incompetence everywhere. 
%% We don't fully understand what we are supposed to develop, we don't fully know what is a good way to do it, and what not. 

%% Every standard and norm is subject of debate and fails in many points. 
%% We learn to compromise the very moment we start our professional career. 
%% Our products are in decay the moment our code base grows. 
%% There is no money, priority to reduce our technical debt. 

%% Write an essay where you analyze how we can use the knowledge from the field of organizational resilience in our field of software engineering.

\chapter{Essay}

Agile methods like Scrum and XP are both based on a few very fundamental assumptions.
They both are very fond of semi-independent, self organising workflows. 
In both processes an assumption is made that the individual is a professional that is able to do its job very well.
So, when a team consists of cunning, socially apt individuals those approaches will probably work very well. 

However, most people are very poor performers. 
And because our field of work is always demanding more people, acceptance interviews are sometimes lacking. 
This results in a lot of people that are not as cunning as one might hope them to be. 
A lack of knowledge or understanding can then result in loss of productivity, or even in a disaster when the amount of skills will deteriorate badly enough inside a team.

So, in order for both Scrum and XP to work, you need a dedicated, professional and cunning team of people.

To what extent would it have solved the problems in the failed project you studied

In the case of the \emph{Betuwelijn}, no real differences would have been made in my opinion. 
Since most of the big failures in the project are related to manipulated numbers from reports, a domain expert would have just behaved in the same way. That person would still be twisting those reports in their favour. 

Another issue in the \emph{Betuwelijn} project was the ever changing scope. Since agile methods even \emph{welcome} this, an even worse outcome might be achieved. The reason I am not sure of this is because in an environment that expects and welcomes change, a more controlled process to these changes might have been started. If this would be the case, I would expect a better outcome. This would be possible because all changes would be critically reviewed by the team and the expert(s) to make sure they make sense.

A trade off that the Agile methods have is their performance. 
While the literature claims high performance gains, empirical studies show that these are almost never achieved, because the balance of all the agile principles need to be met in a perfect manner. This is hardly ever the case. 

Another shortcoming is the lack of direction. 
When a lot of freedom is given to the team to decide on the spot, wrong decisions can be made. A domain expert usually cannot foresee the importance of software related decisions and assumptions. 
This can result in a strained development process, in which a domain expert used to ``just go along'' with a team because he did not understand the magnitude of decisions. 

A last trade off is the lack of out of the box thinking. Domain experts do have a lot of knowledge in their heads, but they also can't use an outside view to approach a problem. 
If a domain expert is very stubborn, a non-optimal solution can be created (see the requirements engineering course).




\bibliographystyle{abbrv}
\bibliography{references}
\chaptermark{Bibliography}

\end{document}
