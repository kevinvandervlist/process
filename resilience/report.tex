\documentclass[]{uva-bachelor-thesis}

% Define this when starting to use the template:
\def\course{Software Process}
\def\assignment{Assignment 6: Resilience}
\def\group{Kevin van der Vlist}
\def\duedate{\today}
% Define your team members hear, each one on a row:
\def\teamtable{\begin{tabular}{ll}
10338152 & kevin.vandervlist@student.uva.nl \\
\end{tabular}}
% The document is now set correctly

\usepackage{graphicx}
\usepackage{wrapfig}
\usepackage{caption}
\usepackage{subcaption}
\usepackage{url}
\usepackage{eurosym}

\usepackage{fancyhdr}
\usepackage[colorlinks, linkcolor=black, urlcolor=black, citecolor=black]{hyperref}
\usepackage{fancyhdr}
\setlength{\headheight}{15pt}

% Chapter and page storage
\def \CurrChapter {}
\def \LastSection {}
\def \CurrSection {}
\newcounter{CurrPage}
\setcounter{CurrPage}{0}

\pagestyle{fancy} 
\renewcommand{\markboth}[2]{\def \CurrChapter {#1}} % For use with \tableofcontents
\renewcommand{\chaptermark}[1]{\def \CurrChapter {#1} \def \CurrSection {}}
\renewcommand{\sectionmark}[1]{ %
  \ifthenelse{ %
    \equal{\thepage}{\value{CurrPage}} %
  } %
  {\def \CurrSection {: #1}} %
  {\let \LastSection \CurrSection \def \CurrSection {: #1} \setcounter{CurrPage}{\value{page}}} %
}

% Headers and footers
\renewcommand{\footrulewidth}{0.5pt}

\fancyhf{}
\fancyfoot[LE,RO]{\textit{\thepage}}
\fancyfoot[RE,LO]{\textit{\group}}
\fancyhead[RE,LO]{\textit{}}
\fancyhead[LE,RO]{ %
  \ifthenelse{ %
    \equal{\thepage}{\value{CurrPage}} %
  } %
  {\textit{\nouppercase{\CurrChapter\LastSection}}} %
  {\let \LastSection \CurrSection \textit{\nouppercase{\CurrChapter\LastSection}}} %
}

\fancypagestyle{plain}{ %
\fancyhf{} %
\fancyfoot[LE,RO]{\textit{\thepage}} %
\fancyfoot[RE,LO]{\textit{\group}} %
\renewcommand{\headrulewidth}{0pt} % remove lines as well
}

\setlength{\parskip}{.5em}

\title{\course}
\subtitle{\assignment}
\author{\group}
\team{\teamtable}

\begin{document}
\maketitle

\setlength{\parskip}{0px}
\tableofcontents
\setlength{\parskip}{.5em}

\clearpage

\chapter{Foo}
% - Organisational resilience paper
% - Resilience for software
% - 

%% Find and read one journal paper about resilience / organizational vigilance. 
%% (Key words: early warning signals, sense making, sense breaking)
%% Find and read one paper that discusses this for software development project.
%% Update your annotated bibliography with interesting facts and thoughts from the papers.

%% The body of research is usually valid for high reliability organizations. These tend to be highly organized, with lot of money spent on problem prevention and training. 
%% These organizations are quite costly and all but lean. 
%% They have the luxury and tendency to be conservative, cautious and accept only high quality solutions. 

%% In software projects it's another world. 
%% There is incompetence everywhere. 
%% We don't fully understand what we are supposed to develop, we don't fully know what is a good way to do it, and what not. 

%% Every standard and norm is subject of debate and fails in many points. 
%% We learn to compromise the very moment we start our professional career. 
%% Our products are in decay the moment our code base grows. 
%% There is no money, priority to reduce our technical debt. 

%% Write an essay where you analyze how we can use the knowledge from the field of organizational resilience in our field of software engineering.

\bibliographystyle{abbrv}
\bibliography{references}
\chaptermark{Bibliography}

\end{document}
